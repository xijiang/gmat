\documentclass[a4]{article}
\begin{document}
\section{Introduction}
$\mathbf{G}$ is being used more and widely.
Its dimension is also growing bigger.
One is because more ID are genotyped.
The other reason is that denser genotype can now be obtained.
Although the hardware prices are going down (really?),
they can't match the exponential growth of the genomic data size.
Suppose there are $n$ genotyped individuals,
each of them has genotypes on $m$ loci.
When GBLUP method is used,
the dimension of G matrix is then $n\times n$.
Since the elements are stored in double precision of 8 bytes,
50k ID

\section{G matrices}
\subsection{VanRaden method 1}
Suppose the SNP genotypes are coded as 0 for (AA), 1 for (AB) and 2 for (BB).
Suppose $p_j$ is the allele frequency observed in the population for locus $j$.
Then the genotypes are corrected by $2p_j$.
\end{document}
